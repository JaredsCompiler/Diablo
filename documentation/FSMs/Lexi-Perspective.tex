\documentclass{article}

\usepackage{hyperref}
\usepackage{indentfirst}

\usepackage{listings}
\usepackage{color}
\usepackage{tikz}

\title{FSM of the Lexical Analyzer (Higher Overview)}

\author{Chris Nutter \\
        Jared Dyreson}

\hypersetup{
    colorlinks,
    citecolor=black,
    filecolor=black,
    linkcolor=black,
    urlcolor=black
}

\lstset{
language=C++,
identifierstyle={\texttt},
moredelim=**[is][\color{red}\ttfamily]{@}{@},
}

\begin{document}

\maketitle
\tableofcontents

\newpage

\section{Overview}

This document explains the inner workings of the Lexical Analyzer from the perspective of the driver code from `lexi.cpp`.
Here, all components designed to make `lexi` a stand alone program work are collated into a single source document.
The structure of this program is as follows:

\begin{enumerate}
\item Source file: where in memory is the file located and then extract the contents of it.
\item Rules: what set of parameters must the lexer abide by to produce the desired tokens.
\item Lexer: the actual mechanism that takes the content of the source document and applies rules to each line, producing tokens if matches are found.
\end{enumerate}

\section{Components of this FSM}

The formal definition of an FSM includes the following:

\begin{enumerate}
\item Finite set of input symbols $\Sigma$: $\{A, B, C\}$ corresponding to the bullet points above.
\item Finite set of states $Q$: $\{D, G\}$
\begin{enumerate}
\item $D$: Unaccepted state. There were no matches and/or lexing errors were thrown.
\item $G$: Accepting state. There was at least one match to the rules presented and no lexing errors thrown.
\end{enumerate}
\item Finite set of accepting states: $F \subseteq Q$
\begin{enumerate}
\item $F \subseteq \{G\}$
\end{enumerate}
\item State-transition function(s) $N: (Q \times \Sigma) \rightarrow Q$
\begin{enumerate}
\item Each constructor and function call will advance the overall mechanism of Lexi
\end{enumerate}
\end{enumerate}

\newpage

\section{Graphical Representation}

Each of the nodes correspond to the defined sets of states and input symbols.

\begin{figure}[!hbtp]
\begin{tikzpicture}[scale=0.2]
\tikzstyle{every node}+=[inner sep=0pt]
\draw [black] (16.9,-18.1) circle (3);
\draw (16.9,-18.1) node {$A$};
\draw [black] (16.9,-35.6) circle (3);
\draw (16.9,-35.6) node {$B$};
\draw [black] (33.9,-25.2) circle (3);
\draw (33.9,-25.2) node {$C$};
\draw [black] (58.4,-12.2) circle (3);
\draw (58.4,-12.2) node {$D$};
\draw [black] (58.4,-35.6) circle (3);
\draw (58.4,-35.6) node {$G$};
\draw [black] (58.4,-35.6) circle (2.4);
\draw [black] (7.3,-22) -- (14.12,-19.23);
\fill [black] (14.12,-19.23) -- (13.19,-19.07) -- (13.57,-19.99);
\draw [black] (19.67,-19.26) -- (31.13,-24.04);
\fill [black] (31.13,-24.04) -- (30.59,-23.27) -- (30.2,-24.2);
\draw [black] (20.2,-35) -- (31.46,-26.95);
\fill [black] (31.46,-26.95) -- (30.52,-27) -- (31.1,-27.82);
\draw [black] (36.55,-23.79) -- (55.75,-13.61);
\fill [black] (55.75,-13.61) -- (54.81,-13.54) -- (55.28,-14.42);
\draw [black] (35.9,-27.6) -- (55.57,-34.59);
\fill [black] (55.57,-34.59) -- (54.99,-33.86) -- (54.65,-34.8);
\end{tikzpicture}
\caption{Lexi FSM}
\end{figure}


\end{document}
