\documentclass{article}

\usepackage{hyperref}
\usepackage{listings}
\usepackage{fancyvrb}
\usepackage{indentfirst}
\usepackage [english]{babel}
\usepackage [autostyle, english = american]{csquotes}
\MakeOuterQuote{"}

\title{FSM of the Lexical Analyzer (Higher Overview)}
\author{Chris Nutter \\
        Jared Dyreson}

\hypersetup{
    colorlinks,
    citecolor=black,
    filecolor=black,
    linkcolor=black,
    urlcolor=black
}

\lstset{
    language=C++,
    identifierstyle={\texttt},
    moredelim=**[is][\color{red}\ttfamily]{@}{@},
} 

\begin{document}

\maketitle
\tableofcontents

\newpage

\section{Regular Expressions}
    The way \emph{Lexi} parses each line and determines the identifier type is through the use of \emph{regular expressions}. Being able to determine the identifier is crucial in defining the token's contents. \emph{Lexi} after processing the file and creating a vector of strings that parses line by line which is then fed through a function that reads each character and determines one of the each lexeme types.

    \begin{enumerate}

        \item \emph{\textbf{Comments}} determines any line that has \emph{!} and ends with a trailing \emph{!}. Multiple comments in a line are supported.
        \begin{Verbatim}
    (!.*!)
        \end{Verbatim}
        
        \item \emph{\textbf{Keywords}} finds any word that is considered reserved for the structure of the language including data types, control-flow operators, and other key-defining words for the language.  
        \begin{Verbatim}
    (int|float|bool|true|false|(end)?if|else|then|while(end)?
    |do(end)?|for(end)?|(in|out)put|and|or|not)
        \end{Verbatim}
        
        \item \emph{\textbf{Number}} is any integer, float, double, size\_t, (etc.) value for identifying amount. 
        \begin{Verbatim}
    (?:b)([-+]?d*.?\\d+)?(?=b)
        \end{Verbatim}

        \item \emph{\textbf{Identifier}} grabs any word that is not within a \emph{comment} or \emph{keyword} field.
        \begin{Verbatim}
    ([a-zA-Z]+(d*)?) 
        \end{Verbatim}
        
        \item \emph{\textbf{Separators}} finds any symbol that helps keep the contents contained.
        \begin{Verbatim}
    (+|-|*|/|=|>|<|>=|<=|&+||+|%|^!$|^)
        \end{Verbatim}
        
        \item \emph{\textbf{Operators}} obtains symbols that the language uses for operation.
        \begin{Verbatim}
    ((|)|{|}|[|]|"|'|,)
        \end{Verbatim}
       
        \item \emph{\textbf{Terminators}} are symbols signalling end of a line.
        \begin{Verbatim}
    (;|$) 
        \end{Verbatim}

    \end{enumerate}

\end{document}
